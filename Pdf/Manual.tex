\documentclass[12pt,a4paper]{article}
\usepackage{fancyhdr}
\usepackage{mwe}
\usepackage{scrextend}
\begin{document}
\pagestyle{fancy}
\fancyhead{}
\fancyhead[R]{Robot Assembly Language User's Manual}
\fancyhead[L]{Polysaga Confidential}
\fancyfoot[C]{\noindent\rule{\textwidth}{0.5pt}\par-\thepage-}

\begin{titlepage}
	\centering
	{\Huge\bfseries Robot Assembly Language\par}
	\vspace{1cm}
	\noindent\rule{\textwidth}{2pt}\par
	\vspace{4cm}
	{\Large\bfseries User's Manual\par}

	\vfill

% Bottom of the page
	{\large \today\par}
\end{titlepage}

\tableofcontents
\pagebreak

% ------------------------- Controls
\section{Controls}

\subsection{Movement}
\begin{labeling}{alligatorssssss}
	\item [\textbf{WASD}] Move player
	\item [\textbf{Mouse}] Rotate camera
	\item [\textbf{Space}] Jump
	\item [\textbf{Ctrl}] Crouch
\end{labeling}

\subsection{Robot Movement}
\begin{labeling}{alligatorssssss}
	\item [\textbf{Arrow Keys}] Move robot arm left/right/forwards/backwards
	\item [\textbf{Home/End}] Move robot arm up/down
\end{labeling}

\subsection{UI}
\begin{labeling}{alligatorssssss}
	\item [\textbf{Tab}] Show mouse cursor
	\item [\textbf{Alt-f4}] Close game (the game will save your progress)
\end{labeling}

\pagebreak

% ------------------------- Instruction reference
\section{Instruction Reference}
This section lists the instructions comprising the Robot Assembly Language.
\pagebreak

\input{instructions/Instructions.tex}

% ------------------------- Device List
\section{Device List}
This section lists the devices that can be accessed by the robot program. These devices can be started/stopped with the \textbf{start} and \textbf{stop} commands, and their memory state can be read with the \textbf{read} command.

All devices have an index. For example, conveyor devices start with \textbf{conveyor0} and go to \textbf{conveyor1}, \textbf{conveyor2}, etc. Similarly, laser devices start with \textbf{laser0}.
\pagebreak


% Conveyor belt
\addcontentsline{toc}{subsection}{conveyor}
\Huge\bfseries conveyor : \large Conveyor belt\par
\noindent\rule{ \textwidth }{2pt}\par
\vspace{ 0.5cm }
\mdseries\normalsize This is the conveyor belt device. \textbf{Start}ing a conveyor belt makes it move items at a rate of 30cm/s. \textbf{Stop}ping a conveyor belt stops it. \textbf{Read}ing a belt does nothing.

\pagebreak

% End effector
\addcontentsline{toc}{subsection}{endeffector}
\Huge\bfseries endeffector : \large End effector\par
\noindent\rule{ \textwidth }{2pt}\par
\vspace{ 0.5cm }
\mdseries\normalsize This is the device on the end of the robot arm. It can grab packages. \textbf{Start}ing an end effector makes it grab packages. \textbf{Stop}ping an end effector makes it release packages. \textbf{Read}ing an end effector does nothing.

\pagebreak

% Laser
\addcontentsline{toc}{subsection}{laser}
\Huge\bfseries laser : \large Laser sensor\par
\noindent\rule{ \textwidth }{2pt}\par
\vspace{ 0.5cm }
\mdseries\normalsize This device shoots a laser and returns 1 if the laser is obstructed. \textbf{Start}ing a laser does nothing. \textbf{Stop}ping a laser does nothing. \textbf{Read}ing a laser returns 1 if the laser is obstructed, 0 if the laser is unobstructed.

\pagebreak


\end{document}